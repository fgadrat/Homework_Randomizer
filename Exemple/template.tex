\documentclass{article}%
\usepackage[T1]{fontenc}%
\usepackage[utf8]{inputenc}%
\usepackage{lmodern}%
\usepackage{textcomp}%
\usepackage{lastpage}%
\usepackage{geometry}%
\geometry{head=40pt,margin=0.5in,bottom=1.0in,includeheadfoot=False}%
%
\usepackage{fancyhdr}%
\usepackage[french]{babel}%
\usepackage{amsmath}%
\usepackage{amssymb}%
\usepackage{graphicx}%
\pagestyle{fancy}%
\fancyhf{}%
\renewcommand{\headrulewidth}{0pt}%
\cfoot{}%
\date{}%
\author{}%
%
\begin{document}%
\normalsize%
\begin{center}
    {\Large\textbf{Devoir maison de {student_name}}}\\
    Géométrie et calculs algébriques
\end{center}

\subsection*{Exercice I}
Dans un repère orthonormé, soient $A\begin{pmatrix}{v1} \\ {v2}\end{pmatrix}$, $B\begin{pmatrix}{v3} \\ {v4}\end{pmatrix}$ et $C\begin{pmatrix}{v5} \\ {v6}\end{pmatrix}$. Soit M le milieu de $[BC]$.

\begin{enumerate}
\item Faire la figure. A compléter au fur et à mesure de l'exercice.
\item Déterminer les coordonnées de M (par un calcul ou par lecture graphique).
\item Calculer les longueurs MA, MB et MC.
\item Que peut-on en déduire pour le triangle ABC ?
\item Déterminer les coordonnées du point D tel que ABDC soit un rectangle.
\end{enumerate}

\subsection*{Exercice II}
\begin{enumerate}
\item Écrire sous forme d'une fraction irréductible : $\frac{{v11}}{{v12}}-\frac{{v13}}{{v14}}$.
\item Développer et réduire les expressions suivantes :
    \begin{enumerate}
    \item $(x+{v15})^2$
    \item $(2x-1)({v16}-x)$
    \end{enumerate}
\item Factoriser les expressions suivantes :
    \begin{enumerate}
    \item ${v17}x^2+{v18}x+{v19}$
    \item $x^2+x(x+{v20})$
    \end{enumerate}
\end{enumerate}
\end{document}
