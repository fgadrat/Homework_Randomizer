\documentclass{article}%
\usepackage[T1]{fontenc}%
\usepackage[utf8]{inputenc}%
\usepackage{lmodern}%
\usepackage{textcomp}%
\usepackage{lastpage}%
\usepackage{geometry}%
\geometry{head=40pt,margin=0.5in,bottom=1.0in,includeheadfoot=False}%
%
\usepackage{fancyhdr}%
\usepackage[french]{babel}%
\usepackage{amsmath}%
\usepackage{amssymb}%
\usepackage{graphicx}%
\usepackage{tikz}%
\usetikzlibrary{calc,positioning}%
\pagestyle{fancy}%
\fancyhf{}%
\fancyfoot[C]{\thepage}
\renewcommand{\headrulewidth}{0pt}%
\date{}%
\author{}%
%
\begin{document}%
\normalsize%
\begin{center}
    {\Large\textbf{Corrigé du devoir de {student_name}}}\\
    Géométrie et calculs algébriques
\end{center}

\subsection*{Exercice I}
\begin{enumerate}
\item Figure:

\begin{center}
\begin{tikzpicture}[scale=0.5]
\draw[gray!50, thin, step=1] (-8,-8) grid (8,8);
\draw[->,thick] (-8,0)--(8,0) node[right]{$x$};
\draw[->,thick] (0,-8)--(0,8) node[above]{$y$};

\foreach \x in {-7,...,7} {
    \draw (\x,2pt)--(\x,-2pt) node[below] {\tiny $\x$};
}

\foreach \y in {-7,...,7} {
    \draw (2pt,\y)--(-2pt,\y) node[left] {\tiny $\y$};
}

\node[label={left:$A$},circle,fill,inner sep=1.5pt] (A) at ({v1},{v2}) {};
\node[label={left:$B$},circle,fill,inner sep=1.5pt] (B) at ({v3},{v4}) {};
\node[label={left:$C$},circle,fill,inner sep=1.5pt] (C) at ({v5},{v6}) {};
\node[label={left:$D$},circle,fill,inner sep=1.5pt] (D) at ({v7},{v8}) {};

\coordinate (M) at ($(B)!0.5!(C)$);
\node[label={left:$M$},circle,fill,inner sep=1.5pt] (M) at (M) {};

\draw (A)--(B)--(C)--(A)--cycle;
\draw (A)--(M);
\draw (C)--(D);
\draw (D)--(B);

\end{tikzpicture}
\end{center}

\item Les coordonnées de $M$ sont : $ M\begin{pmatrix} \frac{x_B+x_C}{2} \\ \frac{y_B+y_C}{2}\end{pmatrix}
=\begin{pmatrix} \frac{{v3}+{v5}}{2}\\ \frac{{v4}+{v6}}{2} \end{pmatrix}
= \begin{pmatrix} {v9} \\ {v10} \end{pmatrix}

\item 
$\begin{aligned}[t]
MA &= \sqrt{(x_A-x_M)^2+(y_A-y_M)^2}\\
&=\sqrt{({v1}-({v9}))^2+({v2}-({v10}))^2}\\
&=\sqrt{({v21})^2+({v22})^2}\\
&=\sqrt{{v23}+{v24}}\\
&=\sqrt{v50}\\
&=5 \sqrt{2}\\
MB &= \sqrt{({v3}-({v9}))^2+({v4}-({v10}))^2}\\
&=\sqrt{({v25})^2+({v26})^2}\\
&=\sqrt{{v27}+{v28}}\\
&=5 \sqrt{2}\\
MC &= \sqrt{({v5}-({v9}))^2+({v6}-({v10}))^2}\\
&=\sqrt{({v29})^2+({v30})^2}\\
&=\sqrt{{v31}+{v32}}\\
&=5 \sqrt{2}\\
\end{aligned}$

\item Comme $MA = MB = MC$, M est donc le centre du cercle circonscrit à ABC. On peut en déduire que le triangle $ABC$ est rectangle en $A$.

\item Par lecture graphique, on trouve que $D\begin{pmatrix} {v7} \\ {v8} \end{pmatrix}$.\\
En calculant : $D=B+\overrightarrow{AC}=\begin{pmatrix} {v3} \\ {v4} \end{pmatrix}+\begin{pmatrix} {v33} \\ {v34} \end{pmatrix}=\begin{pmatrix} {v7} \\ {v8} \end{pmatrix}$

\end{enumerate}
\newpage
\subsection*{Exercice II}
\begin{enumerate}
\item $\begin{aligned}[t]
\frac{{v11}}{{v12}}-\frac{{v13}}{{v14}} &= \frac{{v11}\times {v35}}{{v12}\times {v35}} - \frac{{v13}\times {v36}}{{v14}\times {v36}} \\
&= \frac{{v37}}{{v38}} - \frac{{v39}}{{v38}} \\
&= \frac{{v37}-{v39}}{{v38}} \\
&= \frac{{v40}}{{v38}}{v41}
\end{aligned}$

\item
\begin{enumerate}
\item $\begin{aligned}[t]
(x+{v15})^2 &= x^2 + 2\times {v15}x + {v15}^2\\
&= x^2 + {v42}x + {v43}
\end{aligned}$


\item $\begin{aligned}[t]
(2x-1)({v16}-x) &= 2x\times {v16} - 2x\times x - {v16} + x\\
&= {v44}x - 2x^2 - {v16} + x\\
&=-2x^2 + {v45}x -{v16}
\end{aligned}$
\end{enumerate}

\item
\begin{enumerate}
\item $\begin{aligned}[t]
{v17}x^2+{v18}x+{v19} &= ({v46}x)^2+2({v46}x)({v47})+({v47})^2\\
&= ({v46}x+{v47})^2
\end{aligned}$

\item $\begin{aligned}[t]
x^2+x(x+{v20}) &= x\times x+x(x+{v20})\\
&= x(x+(x+{v20}))\\
&= x(2x+{v20})
\end{aligned}$

\end{enumerate}
\end{enumerate}
\end{document}